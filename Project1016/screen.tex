\documentclass[a4paper]{article}

\usepackage[english]{babel}
\usepackage[utf8]{inputenc}
\usepackage{amsmath}
\usepackage{graphicx}
\usepackage[colorinlistoftodos]{todonotes}

\title{Screens on Screen on Screens}

\author{Matthew W. Flickner}

\date{\today}

\begin{document}
\maketitle

\begin{abstract}
\end{abstract}

\section{Introduction}

talk about that book

\section{Background/Prior Work}
\subsection{Origin of the Multiple Monitor}
While the computer monitor has been an essential part of the modern computer for some time, computers utilizing multiple monitors has been a more development in the past twenty years. The ability of computers to be able to support multiple monitors has been for around since 1998 on Windows systems and 22 years on the Macintosh system.\cite{Grudin}  Since then, technology has evolved to allow numerous monitors to connect to a single computer.

\subsection{Prior Work}
Since the introduction of technology allowing computers to support multiple monitor, researchers have conducted many usability studies studying the efficiency, learnability, and effectiveness comparing users using multiple monitors with users with a singular monitor.

\subsubsection{Grudin's Study}
Jonathan Grudin's study compared the multiple monitor setup upgrade with a larger monitor upgrade and was a huge supporter of the multiple monitor set up. He compared using multiple monitors with a single large monitor to having a house with one large room or with multiple rooms. One could theoretically have a house with one large room that can be used for all purposes. In that one-roomed house there could be everything a person needs to get by. There could be a kitchen, a living room, a work study, and a bed. But it would be far more convenient according to Grudin to have the house with multiple rooms, with each room serving a particular purpose.\cite{Grudin} Grudin's experiment of testing multiple monitors against a single monitor yielded several important observations. Grudin observed that users using a second monitor did not use it as additional space by straddling a window across the two monitors.\cite{Grudin} Grudin also observed that one monitor became the "primary monitor" and was used for the main task while the second monitor was used primarily for secondary tasks to support the primary task.

\subsubsection{Truemper's Study}
This study took a lot of what Grudin did and expanded on it. In addition one important observation that game with this study was the usage of more than just two monitors. The study used four monitors and the amount of time spent using just two windows was far greater than the time using three windows which was far greater than the time spend using four windows.

\subsection{Previously Established Positives of Multiple Monitors}

\subsubsection{Window Visibility}
The multiple monitor setup created a larger average window visibility.\cite{Hutchings} This lessened the need for window minimization and maximization with users.\cite{Grudin} With a larger working space users could spread out their windows to work more comfortably.\cite{Truemper} However the average window visibility was not significant larger than single monitors although this is mostly attributed to multiple monitor users using larger sized windows.\cite{Hutchings}

\subsubsection{Improved Efficiency of Tasks}
Mutliple monitors for one thing definitely improve the efficiency of their users. Users with multiple monitors had better productivity due to lower task completion time and workload.\cite{Kang}

\subsubsection{Bezels}
Bezels are areas between screens.\cite{Truemper} Basically they are the borders of a screen on an individual monitor. When screens are put together in a multiple monitor setup, bezels are useful for supporting multitasking and separating windows with different tasks.



\subsection{Previously Established Negatives of Multiple Monitors}

\subsubsection{Distance to Travel}
With multiple monitors, the distance to travel across the screen is a large factor. Truemper cites the issue of when using the Windows operating system, the user must travel a further distance to reach the start menu in the bottom left corner.\cite{Truemper} This has since been resolved with a keyboard button that brings up the start menu when pushed. However the overall issue of distance to travel is still a issue for other more general scenarios.\cite{Truemper}

\subsubsection{Bezels}
Bezels have a negative side to them as well. They create a visual discontinuity.  When a cursor crosses the path of a bezel, it's path is often deflected. It is possible to calibrate monitors to avoid this issue but the issue then is still a gap between pixels where the bezels exist.\cite{Truemper} 

\subsubsection{Too Many Monitors}
Studies have shown that after a certain amount of screens, some screens are barely used. Truemper's study used four monitors and the amount of time spent using just two monitors was far greater than the time using three monitors which was far greater than the time spend using four monitors. All four monitors were used less than two percent of the time.\cite{Truemper}

\subsubsection{Lack of Application Support}
Many applications do not optimize their software to be used efficiently with a multiple monitor setup.\cite{Grudin}


\section{Method}
The prevailing option at the moment is that a multiple monitor setup is more efficient and productive than a single monitor setup. Because of this, multiple monitor setup are becoming more and more prevalent in work environments and will continue to become more of a norm. The studies of Grudin, Truemper, Kang, and Hutchings are all very relevant studies on the topic and separated by a few years which gives good insight into how usability in multiple monitors has changed as technology has improved over the years. Grudin's study is the basis for many studies to follow and Truemper and Kang are more recent studies. I will use the positive and negative results of a multiple monitor setup from previous works and usability concepts discussed in class along with the psychology of laterality presented by Dr. Hellige to analyze the effectiveness of multiple monitors, singular monitors, and monitor size and draw a conclusion on the most effective way to utilize a computer monitor system.

\section{Discussion}

\subsection{Fitz's Law}

\section{Conclusion}

\bibliography{references}
\bibliographystyle{plain}
\end{document}