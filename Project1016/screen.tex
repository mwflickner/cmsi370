\documentclass[a4paper]{article}

\usepackage[english]{babel}
\usepackage[utf8]{inputenc}
\usepackage{amsmath}
\usepackage{graphicx}
\usepackage[colorinlistoftodos]{todonotes}

\title{Screens on Screen on Screens}

\author{Matthew W. Flickner}

\date{\today}

\begin{document}
\maketitle

\begin{abstract}
\end{abstract}

\section{Introduction}

talk about that book

\section{Background/Prior Work/Review}
\subsection{Origin of the Multiple Monitor}
While the computer monitor has been an essential part of the modern computer for some time, computers utilizing multiple monitors has been a more development in the past twenty years. The ability of computers to be able to support multiple monitors has been for around since 1998 on Windows systems and 22 years on the Macintosh system.\cite{Grudin}  Since then, technology has evolved to allow numerous monitors to connect to a single computer.

\subsection{Prior Work}
Since the introduction of technology allowing computers to support multiple monitor, researchers have conducted many usability studies studying the efficiency, learnability, and effectiveness comparing users using multiple monitors with users with a singular monitor. Jonathan Grudin's study compared the multiple monitor setup upgrade with a larger monitor upgrade and was a huge supporter of the multiple monitor set up. He compared using multiple monitors with a single large monitor to having a house with one large room or with multiple rooms. One could theoretically have a house with one large room that can be used for all purposes. In that one-roomed house there could be everything a person needs to get by. There could be a kitchen, a living room, a work study, and a bed. But it would be far more convenient according to Grudin to have the house with multiple rooms, with each room serving a particular purpose. \cite{Grudin}.


\subsection{History}

\section{Method}
\section{Discussion}
\section{Conclusion}

\bibliography{references}
\bibliographystyle{plain}
\end{document}